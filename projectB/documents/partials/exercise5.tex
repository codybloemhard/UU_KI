\section{Exercise 5}
\subsection{ex5-a}
For the first agent: it's intended behaviour is to find the shortest path to the goal, because there is no
reason not to cause there are no obstacles, therefore every path costs the same. 
It is achieved by sorting the currently visible paths based on their cost so far, with the cheapest sorted 
at the front. That way the algoritm will constantly pick the cheapest path to further explore. 
And because every path costs the same it basically works like breadth first search.
For the second agent: it's intended behaviour is to take the east route because that path has some food on it.
The food makes for a cheaper path than a normal path. It is achieved almost the same way as the first agent, 
only now not every path costs the same. The path with the food costs less, so that path will be explored 
first instead of the other paths.
For the third agent: it's intended behaviour is to take the most west path, because the other paths contain 
ghosts which drive up the cost of a path. This is because the paths with the ghosts have a high chance of death.
It is achieved the same way as with the other agents, only now the paths with the ghosts have a high cost and 
thus they will be sorted at the end. This way the algorithm explores the path without ghosts first.

When the algorithm finds a path to the goal it stops looking because it chose the cheapest path at all time,
therefore the first path must be the cheapest.


\subsection{ex5-b}
