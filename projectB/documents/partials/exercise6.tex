\newpage{}
\section{Exercise 6}
\subsubsection{Results}
\begin{itemize}
\item [\textbf{A*}] score of 456, 535 nodes expanded, cost of 54, found in 0.1 seconds, chooses down over left.
\item [\textbf{DFS}] score of 212, 576 nodes expanded, cost of 298 in 0.0 seconds, goes as far left then down, then as far right and down.
\item [\textbf{BFS}] score of 456, 682 nodes expanded, cost of 54, found in 0.0 seconds, chooses down over left.
\item [\textbf{UCS}] score of 456, 682 nodes expanded, cost of 54, found in 0.0 
seconds, chooses down over left.
\end{itemize}

\subsubsection{Behaviour}
\begin{itemize}
\item [\textbf{BFS}] searches on a level to level basis, it first looks at every path with length k, if it has not reached the goal 
it will expand every path in every possible way that has not been visited yet and gets new paths of length k+1.
\item [\textbf{DFS}] searches on a path to path basis, it tries to extent a path as far as possible, constantly going in the same direction
and only when it can't do that any longer it will try a different direction or go one step back and try a different direction there.
\item [\textbf{UCS}] has a list with paths he has found till that point. It takes the path with the lowest cost, removes it from the list 
and expands that path in every direction, then it calculates the cost for these new paths and puts these paths in the list. 
Then it takes the path with the currently lowest cost and keeps repeating this till it finds the goal.
\item [\textbf{A*}] works like UCS, only doesn't just look at the cost so far, but it looks at a combination of the cost so far and the heuristic.
This heuristic is an estimate of how much it will cost from that point to the goal.
\end{itemize}

\subsubsection{Differences and similarities}
The difference in nodes expanded between BFS and the other is because BFS will also search in the lower right corner. 
This happens because BFS does not know that it is useless to look here since the goal is in the lower left corner.\\
BFS is in this way a dumb algorithm and just searches in every direction it can, but it does always find the shortest path.
The terrible performance of DFS and the number of nodes expanded in this case is mainly caused by which direction it tries first. 
The order it expands is most likely Left, Right, Down and Up. This causes Pacman to go from right to left 
and left to right instead of going straight down. The algorithm will thus first try the path of only left, 
till it hits a wall and then right is not an option so it goes down one, and then continues right till it gets stuck in the lower 
right corner of the upper left "chamber" in the maze. Then after that it will go to just before it enters that "chamber" 
and goes down and tries Left, Right, Down in that order. And that causes it to expand a lot of useless nodes. You could change the 
order in which DFS tries to find a path but then it will perform terrible in other mazes. As follows from the cost, DFS does not 
find the shortest path.\\
UCS works just like BFS because the cost of every move is the same and therefore the priorityqueue of UCS is the same as the queue of BFS.\\
A* expands the fewest nodes because it does not try every possible path. Because the goal is Left-Down from the start state
it will only go left and down, and that is why it does not have to expand as many nodes.\\
A*, UCS and BFS all find the same path because they try to go down before they go left. \\
There is in this example not one shortest path, the algorithm could choose to go left first, then all the way down to the bottom
and then left to the goal and this would give the same cost. If the order of Left, Right, Down, Up was random at every level of BFS
it would probably get a path with a lot of 90 degree turns instead of a few straight lines like it gets now, 
but this would not affect the cost of the path it finds.

