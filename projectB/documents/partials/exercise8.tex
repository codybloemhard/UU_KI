\section{Exercise 8}
\subsection{ex8-a}
Goal-directed task or goal oriented behaviour.

\subsection{ex8-b}
Yes, assuming the terminal state gives a positive reward it is strongly preferred to reach
that terminal state as soon as possible. If you look at planning a couple of actions, you would
like to perform as few actions as possible. This preferred behaviour can be stimulated by 
using a bigger discount, this way one extra action will lead to a significant lower reward,
and therefore the algorithm will choose the plan with the fewest actions. If there is not 
even a single terminal state, the use of a discount is absent. The algorithm can take action
after action for infinite time and so the total reward will be infinitely great.