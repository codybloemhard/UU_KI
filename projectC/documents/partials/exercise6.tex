\section{Exercise 6}
\subsubsection{ex6-a}
By limiting weight change extreme outliers can not influence the weights as much. By not limiting the weight change
large errors may cause weights to overshoot the minima which causes the weights converge slower. Because there is a large
number of features, they may vary a lot in values which may cause such a slowdown.
// TODO improve answer (many features)

\subsubsection{ex6-b}
Autotune trains the model with different values for constant C and picks the best value which achieves the best
accuracy.\\
Autotune has used three different values for C {0.002, 0.004, 0.008}. Between those value of 0.004 for C has given the
highest accuracy on the validation set. \\
MIRA has given 68.0% on the validation and 62.0% on training datasets with C=0.004 whereas perceptron has gives 55.0%
for validation and 48.0% for training dataset. Naive bayes gives 74.0% on the validation and 65.0% on training datasets
with k=0.1.\\
MIRA has better performance than perceptron but performs worse than best naive bayes. \\
MIRA performs better than perceptron because it limits the learning rate and tries to put a margin between the
boundaries which gives it an advantage at classifying unseen data points. \\
MIRA itself is still perceptron based and does not handle inseparable data. Classify digits with pixels as features is
an example of inseparable data and therefore causes MIRA perform worse than naive bayes.

\subsubsection{ex6-c}
The first thing to notice when comparing most relevant weights from perceptron and mira is that more dense regions on
the perceptron are even denser in mira and less dense regions are less dense at MIRA. This is mostly noticable with
digits like 7 and 5.\\
The only difference between the two algorithms is the learning rate. MIRA adds a variable learning rate which tries to
add a margin between between the boundaries. This is responsible for making regions more or less dense than they are on
the perceptron.