\section{Exercise 7}
\subsection{ex7-a}
No naive bayes classifier is not a suitable instrument for classifying digits because it assumes the values of the
features are independent of the values of any other features given a class variable. This is not true because digits
aren't made of independent pixels. They are made of lines / curve segements which together form a digit. Therefore it
is possible that values of features are dependent on other features.

\subsection{ex7-b}
Given command shows all the features which are more likely to belong to label 4 in comparison to label 2. -1 and -2 take
as argument the labels that need to be compared. Since features belong to pixels, these features are drawn again as
pixels/characters. The likelyness is represented by the weights. The label that has a higher weight for a feature in
comparison with another, then it is more likely to represent that label.

\subsection{ex7-c}
In the following list we describe a few features which would have a positive impact at classifying digits. Some of these
are non binary or take up multiple binary inputs.
\begin{description}
\item[Hole count] Some digits have holes. Some have multiple. For example if a digit has 2 holes, then is is quite
certain that it is an 8 since only 8 has 2 holes. We can count holes by partitioning image into multiple groups of
empty/black pixels which are in direct contact. If we reduce this number by once, since background doesnt count as a
hole. Then we can map it into 3 binary features (has no holes, 1 hole, 2 or more). This feature is implemented in our
code and has increased the performance.

\item[Curvature/Convexity defects] Almost all digits are concave except zero. One can fit a polygon around the borders of white pixels
and reduce it to have fewer polygons. This way one can determine the count of convexity defects in the shape using
gift wrapping algorithm. Then this can be stored in multiple binary features (Ex. 0, 1, 2, 3 convex defects). This way
a number can be determined quite accurately. 0 has 0 defects, 1 has 1 defect 2 has 2 defects, 3 has 3 defects, 4 has
3/1 defects and 9 has 1 defect.

\item[Color cross rate] One can count how many times pixel color changes (white -> black) in a row or column and add 3
binary features for each row/column representing 0, 1 or 2 color changes. This is another way to determine if the input
has holes. Since 9 would have highest values in top rows meanwhile 6 would have those in lower rows.

\item[Density] One could split the shape into multiple smaller squares and calculate the density/amount of white pixels
in the given region. Then represent each square in as a binary feature. One could use 0 als "not dense" or "less than 50% is filled"
while 1 gives the opposite. Some digits have high density in their corners. These corners have different positions for every digit.
\end{description}

\subsection{ex7-d}
We have implemented the first described feature. We have done this by taking each pixel not in closed list and expanding
it in all directions using depth first search. Neighbours that are out of bounds or are a white pixel are not added.\\
This has increased the accuracy on the test set from 78\% to 84\% which is quite an improvement and it makes sense since
this is a string indicator which type of digit it is. Digit 8 can be easily classified since its the only that has 2
holes. Also numbers 1, 2, 3 cant be mistaken for other numbers since they dont have any holes.
