\section{Exercise 2}
\subsection{ex2-a}
\begin{table}[]
\begin{tabular}{|l|l||l|l||l|}
\hline
Classifier    & data   & Validation & testing & k \\ \hline
Most frequent & digits & 14\%       & 14\%    & x \\ 
              & faces  & 56\%       & 53\%    & x \\ \hline
Naive Bayes   & digits & 69\%       & 55\%    & 2 \\ 
              & faces  & 77\%       & 75\%    & 2 \\ \hline

\end{tabular}
\end{table}

What is important to know is that for digits there are ten possible labels and for faces there
are only two possible labels. This means that if you randomly guess you will have a higher chance
of guessing correctly with the faces data set. This is what happens for the most frequent 
classifier. The classifier found the most common label in the training set and apparently 
this label also appears with an above average frequency in the validation and testing set.\\
Therefore the scores for digits is 14\% instead of 10\%, which would be the case if all labels
were equally common in the test and validation set. And the same is true for faces, if both 
labels were equally common in the test and validation set, then the scores would be 50\%, 
instead of 56\% and 53\%.\\
The naive bayes classifier scores higher because it really looks at the image. It tries to
find good features that really help distinguish between the different labels.
The score for faces is higher for naive bayes because with the digits the different images will
overlap, even though they are not necessarily the same label. This is because multiple digits 
have, for example, a curve in the top. And with faces it is easier to find a feature that
distinguishes well, because there are only two possible options.

\subsection{ex2-b}
Imagine we have 5 possible labels, these labels do not occur with the same frequency, but rather
one label is quite rare. Then we can have a big training set where this label does not occur. 
And so the classifier will assign a probability of 0 to it. Then when we come across an input
in our test data with this label, the classifier can not possibly assign the correct label.
In order to prevent this, and make sure that every label has a chance to be assigned, we use 
laplace smoothing. 
This entails that we will always add a  standard value to the frequency of a label. 
The probability used to be: frequency of a label divided by the total number of elements 
in the training data. But with laplace smoothing it becomes: (frequency of a label + 1) /
 (number of elements + number of labels).

\subsection{ex2-c}
\begin{table}[]
\begin{tabular}{|l|l||l|l||l|}
\hline
Classifier    & data   & Validation & testing & k \\ \hline
Naive Bayes   & digits & 74\%       & 65\%    & 0.1 \\ 
              & faces  & 85\%       & 82\%    & 0.05 \\ \hline
\end{tabular}
\end{table}

The autotune option finds the best value for k among these [0.001, 0.01, 0.05, 0.1, 0.5, 
1, 5, 10, 20, 50]. And then the classifier uses the best k value that was found for the
validation and test set. 
This option helps to improve the score for the naive bayes classifier and will in general 
always give better scores. It is better for both data sets, but there could be a dataset 
where a k of 2 would give the best scores and there the autotune would give worse scores.
What does stay the same is the presence of a difference between accuracy on validation and 
testing set. Autotune will always choose a k that performs best on the validation set, 
but this is not necessarily the best k for the test set.

\subsection{ex2-d}
For naive bayes: first it gets the classifier by looking at the training set, then the 
validation set can be used to get a good k value and make sure we do not
get a classifier that does really well on the training set but poorly on other sets.
And as last it is tested on the test set to get an indication of how the classifier will
perform on real life data sets.
For  most frequent: it gets the classifier by looking at the training set. It looks which label
is the most common and then uses that as classifier. It does nothing important with the validation set.
It merely looks what score it gets on there. Then it uses the test set to get an indication 
of how the classifier will perform on real life data sets.


\subsection{ex2-e}
The test set is used to het an indication of the expected performance in the real world. This is because the data in the test set is not used for training. This means it is not overfit on the test set and you can see how well the classifier generalizes. This is import, for it will receive data it has not been trained on either in the real world. The validation set can be use to fine tune the classifier and rule out any
form of overfitting. This happens with the naive bayes classifier when the autotune option enable. 
The validation test is used to get the best k value from the training data.
