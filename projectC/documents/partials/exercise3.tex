\section{Exercise 3}
\subsection{ex3-a}
Because there are a lot of features which vary between zero and one multiplying them together can yield very small
numbers. These numbers in some cases cannot be represented using float and double types.\\
A workaround for this is representing probabilities a log probabilities. This way their values are between -\infty and 0.\\
Also because of this we no longer use product, but use summation instead. This works since sum of logarithms is
equivalent to the log of the product of all the probabilities.

\subsection{ex3-b}
Conditional probability can also be written as P(A^B) = P(A|B)P(B) . With this we can see that joint probability yields
the same result as posterior probability multiplied with probability of P(B). \\
Using log in joint probabilities achieves same result as described earlier because sum of logarithms of the
probabilities is equivalent to the log of the product of all the probabilities.
//
TODO: I could be wrong here at my first point. Doublecheck

\subsection{ex3-c}

\subsection{ex3-d}
