\section{Exercise 3}
\subsection{ex3-a}
Because there are a lot of features which vary between zero and one multiplying them together can yield very small
numbers. These numbers in some cases cannot be represented using float or double types.\\
A workaround for this is representing probabilities as log probabilities. This way their values are between -$\infty$ and 0.\\
Also because of this we no longer use product, but use summation instead. This works since sum of logarithms is
equivalent to the log of the product of all the probabilities.

\subsection{ex3-b}
Conditional probability can also be written as P(A$\wedge$B) = P(A|B)P(B) . With this we can see that joint probability yields
the same result as posterior probability multiplied with probability of P(B). \\
Using log in joint probabilities achieves same result as described earlier because sum of logarithms of the
probabilities is equivalent to the log of the product of all the probabilities.

\subsection{ex3-c}
\begin{table}[!htbp]
\centering
\begin{tabular}{l|llllll}
                    & 0.55   & 0.60   & 0.65   & 0.70   & 0.75   & 0.80   \\ \cline{2-7}
accuracy test       & 85.0\% & 85.0\% & 85.0\% & 85.0\% & 85.0\% & 85.0\% \\ \cline{2-7}
accuracy validation & 81.0\% & 81.0\% & 81.0\% & 81.0\% & 81.0\% & 81.0\%
\end{tabular}
\end{table}
Thresholding does not seem to to improve performance. It rather worsens the performance by introducing more false
negatives. The posterior probabilities are all around 50\% for which thresholding by the given amounts is too much. This
is because features are of a low quality.//

Reproduce dataClassifier.py -c naiveBayes -d faces -k 0.1 --threshold=0.45,0.55

\subsection{ex3-d}
The naive bayes classifier would be more biased towards classifying numbers as even numbers. Normally one would need to
balance the dataset first, but this would require retraining. Since naive bayes works using frequency counting we can
also correct the amount of even number occurences in the frequency table by reducing them by 50\%. This would balance the
dataset out.